# PE-Style Carve-Out Investment Committee Memorandum  
## GE Aerospace  

**Prepared by:** Harshit Singh  
**Date:** 2026-01-06  

---

## 1. Executive Summary

**IC Decision:**  
Evaluate GE Aerospace as a standalone, high-quality industrial asset post-separation, suitable for PE-style underwriting based on durable services cash flows, strong backlog visibility, and conservative capital structure, subject to execution risks associated with separation, transition services, and legacy obligations as disclosed.

---

### Information Boundary and Methodology

This memorandum is intentionally constrained to publicly disclosed information available through SEC filings, investor materials, and separation-related agreements. Where key economic inputs (including TSA pricing, quantified dis-synergies, pension magnitude, or separation cost phasing) are not publicly disclosed, they are treated as execution risk rather than modeled inputs.

No attempt has been made to infer, proxy, or normalize non-disclosed data.

---

### Transaction Overview

General Electric Company completed the separation of its energy businesses into a newly listed entity, GE Vernova Inc., on April 2, 2024, via a tax-free pro rata distribution to GE shareholders.

Following the spin-off, General Electric Company now operates exclusively as **GE Aerospace**, comprising commercial engines and services and defense and propulsion technologies.

The separation was effected pursuant to a Form 10 registration statement and a series of intercompany agreements, including:

- Separation and Distribution Agreement  
- Transition Services Agreement  
- Tax Matters Agreement  
- Employee Matters Agreement  
- Related governance documents  

GE Aerospace is now a standalone public company, while certain transitional and commercial relationships with GE Vernova continue under these agreements.

The spin-off is intended to qualify as a tax-free transaction for U.S. federal income tax purposes under Sections 355 and 368(a)(1)(D) of the Internal Revenue Code, subject to ongoing compliance with post-distribution restrictions.

---

### Business Overview

GE Aerospace operates a global aviation franchise focused on the design, manufacture, and servicing of aircraft engines and propulsion systems for commercial and defense applications.

The business is characterized by:
- A large installed base of engines  
- Long-term service agreements  
- Recurring aftermarket revenue streams  

As disclosed in post-separation filings, services revenue represents a substantial portion of segment revenues, supported by a significant reported remaining performance obligation (RPO) backlog that provides multi-year revenue visibility.

Revenue and profitability are driven primarily by utilization of the installed base rather than new engine deliveries alone.

The company’s customer base includes:
- Global airlines  
- Aircraft lessors  
- Defense customers  
- Government-affiliated entities  

Operations span manufacturing, maintenance, repair, and overhaul activities across multiple geographies.

---

### Investment Thesis

The investment case for GE Aerospace as a PE-style standalone asset rests on the following disclosed characteristics:

1. **Durable, Contracted Services Cash Flows**  
   GE Aerospace’s reported backlog and installed engine base underpin long-duration service revenues that are less cyclical than original equipment sales. Management disclosures emphasize the importance of aftermarket services in driving profitability and cash generation.

2. **Post-Separation Business Focus and Operational Clarity**  
   The completion of the GE Vernova spin-off simplifies the corporate structure and isolates aerospace operations, allowing capital allocation and management focus to be directed exclusively toward aviation-related priorities.

3. **Conservative Capital and Liquidity Profile**  
   Public filings disclose a committed revolving credit facility, defined leverage calculations, and liquidity provisions designed to support standalone operations. The company has disclosed positive free cash flow generation in recent periods following the separation.

No reliance is placed on margin expansion, multiple re-rating, or growth acceleration beyond disclosed trends.  
The investment case rests on durability and execution rather than upside transformation.

---

### Execution and Separation Considerations

GE Aerospace continues to operate under a Transition Services Agreement (TSA) with GE Vernova, pursuant to which each party provides certain transitional corporate services to the other.

While the existence, governance framework, and cost-based nature of these services are disclosed, the following are **not publicly disclosed**:
- Service-by-service pricing  
- Duration  
- Total cash cost schedules  

The company has incurred and continues to incur separation-related costs associated with:
- Systems implementation  
- Business and facilities separation  
- Employee-related matters  

Historical separation costs have been disclosed in periodic filings, but management has not provided:
- A consolidated forward estimate of total remaining separation costs  
- A detailed cash vs non-cash breakdown  

Pension and other post-retirement benefit obligations have been allocated among GE Aerospace, GE Vernova, and GE HealthCare based on employee legacy business affiliation.

While the allocation mechanism and responsibility for ongoing funding are disclosed, detailed dollar allocations and projected funding schedules specific to GE Aerospace are not publicly disclosed.

---

### Key Risks

Key risks disclosed in connection with the separation and standalone operations include:

- **Transition Execution Risk**  
  Potential inefficiencies, increased costs, or operational disruption as GE Aerospace exits transitional service arrangements and establishes fully independent systems and processes.

- **Cost Dis-synergies**  
  Loss of shared services, scale benefits, and corporate infrastructure previously provided by GE, which may result in higher standalone operating costs than anticipated.

- **Pension and Legacy Obligations**  
  Exposure to changes in actuarial assumptions, asset performance, and funding requirements related to defined benefit pension and post-retirement plans.

- **Regulatory and Contractual Constraints**  
  Ongoing compliance with tax-free spin-off restrictions and other contractual covenants that limit certain strategic actions during the restricted post-distribution period.

---

### Preliminary Assessment

Based solely on publicly disclosed information, GE Aerospace represents a high-quality industrial asset with strong services visibility and improving standalone financial profile.

However, a PE-style underwriting must explicitly account for the absence of:
- Publicly disclosed TSA economics  
- Quantified dis-synergies  
- Detailed separation cost phasing  

Further diligence beyond public disclosures would be required to fully assess normalized cost structure, cash flow sustainability post-TSA, and the long-term impact of legacy obligations before making a definitive investment recommendation.

---

## 2. Standalone Operating Model

This section outlines the standalone operating characteristics of GE Aerospace based exclusively on publicly disclosed information following the completion of the GE Vernova separation.

No normalization adjustments, forward projections, or inferred assumptions are applied beyond what is explicitly disclosed.

---

### Revenue Composition

GE Aerospace reports revenue across two primary operating categories:
- Commercial Engines & Services  
- Defense & Propulsion Technologies  

A substantial portion of total revenue is derived from aftermarket services associated with the installed base of engines.

Services revenue is supported by long-term contractual arrangements and a significant remaining performance obligation (RPO) backlog.

However, the following are **not publicly disclosed**:
- Contract duration by program  
- Pricing escalation mechanics  
- Service margin profile  

Revenue visibility is therefore qualitative rather than quantitatively modeled.

---

### Cost Structure

Public filings describe GE Aerospace’s cost structure as consisting primarily of:

- Manufacturing and production costs  
- Service delivery costs (MRO)  
- Engineering, research, and development  
- Selling, general, and administrative expenses  

Post-separation, GE Aerospace incurs incremental standalone costs previously borne at the GE consolidated level.

While management acknowledges these costs, no quantitative estimate of incremental standalone SG&A is publicly disclosed.

---

### Transition Services and Operational Independence

GE Aerospace and GE Vernova operate under a TSA with:
- Steering committee oversight  
- Cost-based pricing framework  

Not publicly disclosed:
- TSA pricing schedules  
- Aggregate annual TSA cash costs  
- Service-specific exit timelines  

As a result, TSA impacts are treated as execution variables, not modeled inputs.

---

### Free Cash Flow Characteristics

GE Aerospace has disclosed positive free cash flow generation post-separation.

Cash generation is attributed to:
- Service revenue  
- Backlog execution  
- Working capital dynamics  

Not disclosed:
- Normalized maintenance vs growth capex  
- Cash sensitivity to utilization or service mix  
- TSA unwind cash impact  

Free cash flow is therefore treated as an observed outcome rather than a forecast.

---

### Working Capital and Contractual Dynamics

GE Aerospace operates with complex working capital dynamics related to:
- Long-term contracts  
- Advance payments  
- Milestone billings  

Not disclosed:
- Contract-level advance structures  
- Deferred revenue amortization  
- Customer-specific exposure  

Working capital is recognized as material but not modeled.

---

### Pension and Legacy Cost Considerations

GE Aerospace retains responsibility for pension and post-retirement benefit obligations.

Disclosed:
- Allocation mechanism  
- Sensitivity to actuarial assumptions  

Not disclosed:
- Standalone pension asset/liability balances  
- Funding schedules  
- Sensitivity quantification  

These obligations represent structural considerations but cannot be quantified.

---

### Summary Operating Assessment

Disclosed standalone characteristics include:
- Revenue visibility from installed base and backlog  
- Transitioning cost structure  
- Positive reported free cash flow  
- Execution risk tied to TSA exit and separation costs  

Durability is supported qualitatively, with uncertainty around normalized economics.

---

## 3. Capital Structure & Liquidity

This section assesses GE Aerospace’s standalone capital structure and liquidity profile based exclusively on public disclosures.

No forward leverage assumptions or refinancing scenarios are applied.

---

### Standalone Capital Structure

GE Aerospace operates with:
- Standalone debt under post-separation credit agreements  
- Public equity capitalization  
- Residual intercompany arrangements  

Not publicly disclosed in a consolidated form:
- Maturity ladder  
- Tranche-level amortization  
- Fixed vs floating rate breakdown  

---

### Credit Facilities and Covenants

GE Aerospace has disclosed committed revolving credit facilities for:
- Liquidity backstop  
- Working capital  
- General corporate purposes  

Not fully disclosed:
- Consolidated covenant thresholds  
- Covenant headroom under downside scenarios  

Covenant risk is acknowledged structurally but not quantified.

---

### Liquidity Sources

Disclosed liquidity sources include:
- Cash and cash equivalents  
- Revolver availability  
- Operating cash flow  

No minimum liquidity policy or downside buffer targets are disclosed.

---

### Cash Flow Deployment Priorities

Disclosed priorities include:
- Operations and capex  
- Debt service  
- Pension funding  
- Shareholder returns (subject to board approval)  

No fixed capital return framework is disclosed.

---

### Separation and Transition Cash Considerations

Not publicly disclosed:
- Total remaining separation cash outflows  
- Year-by-year cash phasing  
- TSA reimbursement or duplication effects  

Separation-related cash impacts remain unquantified.

---

### Pension and Long-Term Obligations

Funding schedules and annual contribution requirements are not disclosed.

These obligations introduce long-term considerations but cannot be modeled.

---

### Liquidity Risk Assessment

Liquidity profile characterized by:
- Multiple liquidity sources  
- No disclosed near-term stress  
- Exposure to unquantified transition-related cash demands  

Liquidity adequacy depends on execution discipline.

---

### Capital Structure Summary

GE Aerospace enters its standalone phase with:
- A functioning capital structure  
- Adequate disclosed liquidity  
- Limited transparency into downside headroom  

Capital structure risk is execution-driven, not leverage-driven.

---

## 4. Downside Case & Capital Protection

This section evaluates resilience under execution-driven downside scenarios.

No macro collapse or demand shock is assumed.

---

### Downside Framework

Risks considered:
- Delayed operational independence  
- Higher separation costs  
- Extended TSA reliance  
- System migration inefficiencies  

---

### Primary Downside Drivers

#### Transition Services Drag
- Prolonged TSA duration  
- Delayed internal systems  
- Duplicate cost structures  

#### Separation Cost Overrun
- Additional system or governance costs  
- Delayed cost savings  

#### Incremental Standalone Cost Structure
- Higher public company SG&A  
- Slower overhead reduction  

---

### Capital Protection Mechanisms

**Business Durability:**  
Installed base and service backlog support resilience.

**Liquidity Access:**  
Cash, revolver capacity, and operating cash flow provide buffers.

**Absence of Structural Triggers:**  
No disclosed near-term maturities or forced refinancing.

---

### Failure Conditions

Capital impairment would require:
- Prolonged TSA dependency  
- Sustained separation cost overruns  
- Inability to offset standalone cost increases  

---

### IC Perspective on Downside Risk

Downside characterized by:
- Execution risk, not demand risk  
- Liquidity pressure, not solvency risk  
- Manageable exposure with disciplined execution  

---

## 5. Final IC Recommendation

Based exclusively on public disclosures, the recommendation is to **approve the GE Aerospace standalone investment thesis**, subject to execution and capital protection conditions.

---

### Investment Rationale Summary

- Large installed base  
- Recurring aftermarket revenue  
- Backlog visibility  
- Positive disclosed free cash flow  
- Multiple liquidity sources  

---

### Risk Acceptance Framework

Approval accepts execution risk but does not rely on:
- Margin expansion  
- Growth acceleration  
- Multiple re-rating  

---

### Capital Protection Conditions

Approval contingent upon:
- Adequate liquidity buffers  
- No emergence of covenant or maturity pressure  
- Progress in TSA unwind  
- Separation costs within disclosed historical ranges  

---

### Disapproval / Reassessment Trigger

Reassessment required if:
- TSA dependence extends materially  
- Separation costs and SG&A outpace cash generation  
- Liquidity compresses on a sustained basis  

---

### IC Decision

The decision reflects controlled risk acceptance:  
Ownership of a resilient industrial franchise is justified only while execution risk remains bounded and capital protection mechanisms remain intact.